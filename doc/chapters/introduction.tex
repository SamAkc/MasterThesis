\chapter{Einleitung} % Kapitel zur Einleitung
\enquote{Die Herausforderungen unserer Zeit sind zu groß, als dass ein Mensch sie alleine lösen könnte} - so lautet ein Auszug der Partei GRÜNE in ihrem Regierungsprogramm\sidenote{\url{https://www.gruene-bayern.de/programm/}} für die Landtagswahl 2023 in Bayern. Mit diesem Slogan bezieht sich die Partei darauf, politische Ziele in Form vom Bau bezahlbarer Wohnungen in Gemeinden, der Planung von Windrädern mit Nachbarorten zu verwirklichen, und das alles auf eine sichere und klimaneutrale Lebensweise. \\ Diese Denkweise kann immer häufiger in der Gesellschaft beobachtet werden, wenn große Unternehmen wie Apple, BMW und IBM, obwohl sie aus unterschiedlichen Domänen stammen, mit demselben Ziel an die Öffentlichkeit treten: der Nachhaltigkeit. Elektronische Geräte wie Smartphones und Notebooks werden mit 100\% recycelten Materialien\sidenote{\url{https://www.apple.com/de/iphone-15-pro/}} hergestellt, während Automobilhersteller auf vollelektrische Automodelle unter der Einhaltung der Klimaneutralität\sidenote{\url{https://www.bmw.de/de/topics/service-zubehoer/bmw-special-sales/sustainability/bmw-special-sales-sustainability-hub-uebersicht.html}} setzen. Die Motivation hinter diesem Umdenken ist der Klimawandel, denn die CO2-Emissionen auf der Welt haben sich seit 1960 von knapp 10.000 Tonnen CO2eq\sidenote{Formelzeichen für das Treibhauspotenzial} auf knapp 40.000 Tonnen CO2eq bis 2021 vervierfacht \cite{GlobalCarbonAtlas2023}. Als Industrieland trägt Deutschland maßgeblich zum Ausstoß der Treibhausgase bei, denn hier kommen weiterhin 53,7\% konventionelle Energieträger\sidenote{Kohle, Kernenergie und Erdgas} im industriellen, aber auch privaten Sektor zum Einsatz \cite{StatistischesBundesamt2023}. Mit dieser Bilanz hat Deutschland 2021 den siebthöchsten Ausstoß an CO2-Emissionen mit 674,75 Tonnen CO2eq im weltweiten Vergleich \cite{GlobalCarbonAtlas2023}. \\
Umso wichtiger ist es also, den Aspekt der Nachhaltigkeit im Zuge des Klimawandels in immer mehr Bereiche des Alltags zu integrieren. Einer dieser Bereiche ist die Softwareentwicklung, denn Software ist in der heutigen Zeit in allen Bereichen des Lebens zu finden, unabhängig davon, ob im privaten Sektor, in der Industrie oder in der Politik. Aufgrund dessen ist es wichtig, dass Softwareentwickler*innen sich mit dem Thema der Nachhaltigkeit auseinandersetzen und Lösungen entwickeln, die diesen Aspekt berücksichtigen.

Motiviert durch die Nachhaltigkeit und den Klimawandel, wird in Bamberg seit Anfang 2022 auf Initiative des \ac{BVM} ein Klimamessnetz\sidenote{\url{https://bvm-bamberg.de/de/projektseiten/klima/}} betrieben, welches die Temperatur und die Luftfeuchtigkeit an verschiedenen Standorten innerhalb der Stadt misst. Die Messungen erfolgen dabei durch smarte Wetterstationen von Netatmo\sidenote{\url{https://www.netatmo.com/de-de/smart-weather-station}} und die Auslesungen können auf einer interaktiven Karte in einer für diesen Einsatz vorgesehenen Webapplikationen\sidenote{\url{https://weathermap.netatmo.com/}} in Echtzeit ausgelesen werden. Das Ziel des Klimamessnetzes ist es laut eigener Aussage, ein grundlegendes Verständnis über das Klima in Bamberg zu erhalten, um Verantwortungspersonen, wie Politiker*innen, Stadtplaner*innen und Ämter, zu erreichen und die Stadt nachhaltiger zu gestalten. \\ Ein Rückschritt beim Erreichen dieses Ziels sind auftretende Fehler und Anomalien in den Auslesungen der smarten Wetterstationen: Durch exogene Einflüsse\sidenote{z.B. Stationen werden verdeckt oder durch reflektierende Fenster direkt von der Sonne angestrahlt, Wärmestrahler von Gaststätten und Restaurants werden in unmittelbarer Nähe eingeschaltet oder Vergleichbares} auf die Stationen treten Ausreißer in den Auslesungen auf, die die Sensordaten, und damit die Repräsentativität der Daten negativ beeinflussen. 

Diese Arbeit soll gleichzeitig eine Lösung für das beschriebene Problem des Klimamessnetzes in Bamberg und einen Beitrag durch die Bereitstellung einer Softwarelösung motiviert durch den Klimawandel bieten. Dabei wird das Klimamessnetz aufgegriffen und durch den Ansatz des Crowdsensings erweitert. Konkret bedeutet dies, dass Bürger*innen der Stadt Bamberg durch den Einsatz eines interaktiven Werkzeugs in der Lage sein sollen, die Umweltdaten insofern zu kuratieren\sidenote{Mit der Kuratierung wird in dieser Arbeit die Tätigkeit bezeichnet, erfasste Daten auf ihren repräsentativen Wert zu bewerten und darauf basierend zu kategorisieren (\textit{lateinisch:} \enquote{sich kümmern um})}, sodass auftretende Ausreißer, Anomalien und Fehler in den Sensordaten auf ein Minimum begrenzt werden können. Mit diesem Prozess wird außerdem präsentiert, wie die Vorgehensweise der Softwareentwicklung durch den Crowdsensing-Ansatz beeinflusst wird. Im Anschluss wird die Softwarelösung als Artefakt genutzt, um die aus dem Crowdsensing-Ansatz resultierenden Vor- und Nachteile auf die Software selbst, aber auch den Einsatz dieser zu evaluieren. Konkret soll die folgende Forschungsfrage beantwortet werden: 

\textbf{Wie plant und entwickelt man eine Softwarelösung zur Kuratierung von Umweltdaten einer bürgerinitiierten Crowdsensing-Kampagne?} 


Um diese Frage zu beantworten, ist die Arbeit wie folgt strukturiert: Zunächst wird der Hintergrund dieser Arbeit präsentiert, um eine Grundlage für die nachfolgenden Kapitel zu schaffen. Auf dieser Grundlage basiert das Konzept, welches die Softwarelösung konzeptuell beschreibt, die im Anschluss in der Implementierung umgesetzt wird. Vor der Implementierung ist es aber erforderlich, die Anforderungen an die Software zu definieren, um die Umsetzung zu erleichtern. Nach der Implementierung wird die Software evaluiert, um die Vor- und Nachteile des Crowdsensing-Ansatzes in der Softwareentwicklung zu präsentieren, aber auch den gesamten Prozess dieser Arbeit zu bewerten. Aus dieser wird ein Fazit gezogen, welches die Frage beantwortet, ob der Crowdsensing-Ansatz in der Softwareentwicklung sinnvoll ist und in zukünftigen Implementierungen berücksichtigt werden soll.