\chapter{Einleitung} % Kapitel zur Einleitung
% Die Motivation hinter meinem Tool, wie kam ich auf die Idee, welches Ziel hat mein Tool, 
% wie gehe ich vor und baue das ganze Projekt im Allgemeinen auf? Welchen Einfluss hat Crowdsensing hierbei?
Umwelt immer wichtiger, Klimawandel, Erderwärmung... Politik, aber auch Unternehmen setzt immer mehr darauf (Statistik hier zB), umso wichtiger ist es, das Thema Umwelt/Nachhaltigkeit 
in immer mehr Bereiche zu integrieren, unter anderem in die Softwareentwicklung. Allerdings fühlen sich immer mehr Bürger von Politik abgehängt, oder nicht 
respektiert/berücksichtigt, also braucht man Hard Facts/Statistik, um vorweisen zu können, was abgeht und welche Maßnahmen erforderlich sind. Was muss beim Städtebau berücksichtigt werden, wo sind die wärmsten 
Bereiche in einer Stadt und warum? Um dies aufzuzeigen, mein Tool als Lösung. Zwei Fliegen mit einer Klappe, da der Aspekt des Crowdsensings in die Softwareentwicklung miteinfließt und so nachgewisen werden kann, wie sich das
auf die Softwareentwicklung auswirkt und ob das überhaupt sinnvoll ist, das zu berücksichtigen. Wenn ja, eventuell in der Zukunft viel größerer Fokus darauf?

\ac{acro} Bürgerverein Mitte mit dem Klimamessnetz in Bamberg, initiiert durch Prof. Foken. Regelmäßige Analysen und Auswertungen, z.T. sogar veröffentlicht, um aufzuzeigen, was falsch läuft in der Stadt. Allgemeines Ziel des BVM ist das Fördern von mehr Grün in der Innenstadt, als auch die 
Reduzierung des Verkehrs. Analysen und Auswertungen erfolgen dabei aber manuell, und 
es kommt regelmäßig zu Anomalien/Fehlern. Hier kann man natürlich irgendwelche qualitätssteigernde Algorithmen drüber laufen lassen, aber das erwischt natürlich auch nicht alles (Verweis auf Reem) - also Einbinden des Crowdsensing-Aspekts, als zusätzliche Qualitätsschicht.

Mehr grün in der Innenstadt, Reduzierung von Verkehr sind Ziele des BVM. Mein Ziel ist das Schaffen einer Schnittstelle zwischen Stakeholder und Sensordaten, um das Erreichen der Ziele der Stakeholder zu unterstützen. Am Ende möchte ich noch zeigen, ob der Aspekt des Crowdsensings irgendwas bringt in der Softwareentwicklung oder ob
das zu vernachlässigen ist, vor allem auf die Langzeit betrachtet. Tool soll Auswertungen unterstützen und bei deren Treffen regelmäßig zum Einsatz kommen.