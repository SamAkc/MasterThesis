\chapter{Diskussion und Evaluation}
Das Ziel dieser Arbeit besteht darin, die Einführung des Crowd\-sen\-sing-Aspekts in die Softwareentwicklung und dessen Auswirkungen auf die Planung, die Software selbst und die Stakeholder zu analysieren. In diesem Kapitel werden die Ergebnisse dieser Arbeit diskutiert und evaluiert. Dabei wird zunächst darauf eingegangen, inwiefern die Anforderungen aus Kapitel \ref{sec:requirements} erfüllt worden sind. \\ Anschließend findet eine Evaluation des Projektes statt, in der zuerst die persönlichen Erfahrungen des Autors mit dem Projekt selbst, den Stakeholdern und anschließend unter Zuhilfenahme der Literatur diskutiert werden. Abschließend werden die Limitationen der Arbeit aufgezeigt.
\label{sec:discussion}

\section{Erfüllen der Anforderungen}
\label{sec:requirements_evaluation}
Da dieses Projekt eine zeitliche Limitierung aufgewiesen hat, ist es grundsätzlich nicht möglich gewesen, alle aufgestellten Anforderungen zu erfüllen. Die Ursache hierfür hat darin gelegen, dass die Umsetzung entweder mehr Zeit oder zusätzliche Beitragende der Arbeit selbst benötigt hatte. Aufgrund dieser Tatsache spielt die Priorisierung eine große Rolle: Diese ermöglicht, eine Reihenfolge der zu erfüllenden Anforderungen einzuhalten und somit die wichtigsten Anforderungen zuerst zu erfüllen. \\ In die Priorisierung fließt gleichzeitig mit ein, den Crowdsensing-Aspekt umzusetzen, sodass die aufgestellte Forschungsfrage beantwortet werden kann. Konkret bedeutet das, dass folgende Anforderungen \textbf{vollständig} umgesetzt worden sind:

\begin{itemize}
    \item \textbf{Einsehen einer interaktiven Karte:} Eine OpenStreetMap-Karte ist in der Startseite der Applikation eingebettet.
    \item \textbf{Vergleiche zwischen Stationen:} Stationen können über ein Dropdown-Menü ausgewählt und miteinander verglichen werden. Dabei werden je nach Auswahl die entsprechenden iFrames auf Grafana ausgewählt und angezeigt.
\end{itemize}

Die folgenden Anforderungen wurden \textbf{teilweise} umgesetzt:

\begin{itemize}
    \item \textbf{Aufbauen einer Datenbank:} Eine grundlegend funktionierende Datenbank ist mithilfe von Docker umgesetzt und kann in die Applikation eingebunden werden. Diese Datenbank ist rudimentär und befindet sich lokal in einem Container auf der Maschine der jeweiligen Nutzer*innen. Ein Datenmodell (z.B. für das User Handling) ist nicht umgesetzt. Ein Hosting in einer Cloud oder einem globalen Server sollte angestrebt werden.
    \item \textbf{Einsehen der Sensordaten:} Die Sensordaten werden über eine parallel laufende Abschlussarbeit manuell mithilfe von exportierten CSV-Dateien in Grafana eingebunden und visualisiert. Allerdings ist die Anzeige der Daten nicht in Echtzeit, sondern nur in Form von statischen Graphen möglich. Eine Anbindung an die Datenbank sollte angestrebt werden.
    \item \textbf{Chats für den Bereich \enquote{Sensorinspektor}:} Grundlegend ist es möglich, Annotationen an den Graphen im Bereich \enquote{Sensorinspektor} hinzuzufügen. Hierbei wird zur jeweiligen Position am Graphen zusammen mit dem Zeitstempel eine Texteingabe eingefügt. Allerdings findet eine Speicherung der Eingabe in der Datenbank nicht statt, sodass beim Aktualisieren der Applikation die Annotationen verloren gehen. Eine Anbindung an die Datenbank sollte angestrebt werden.
    \item \textbf{Chats für den Bereich \enquote{Meine Karte}:} Es befindet sich ein Chat-Fenster neben der Karte im Bereich \enquote{Meine Karte}. Hier können Nutzer*innen Chats einfügen, die im Anschluss im Chatfenster angezeigt werden. Allerdings werden diese Chats weder in einer Datenbank gespeichert, sodass bei einer Aktualisierung der Seite alle Chats verloren gehen, noch werden diese mit Daten verknüpft. Andere Nutzer*innen sind ebenfalls nicht in der Lage, die Chats einzusehen. Eine Anbindung an die Datenbank sollte angestrebt werden.
    \item \textbf{Einbinden der Wetterstationen:} Die Wetterstationen sind mithilfe von Google Earth Pro auf der Karte markiert. Der nächste Schritt ist das Exportieren von KML\sidenote{Keyhole Markup Language: Dateiformat zum Darstellen von geoografischen Daten (vgl. \url{developers.google.com/kml})} -Dateien aus Google Earth Pro, um diese mithilfe von OpenLayers in die OpenStreetMap-Karte einzubinden\sidenote{\url{openlayers.org/en/latest/examples/kml.html}}. Allerdings ist dieser Aspekt aus zeitlicher Sicht nicht mehr umsetzbar, sodass die Stationen nicht auf der Karte angezeigt werden.
    \item \textbf{\ac{LCZ} auf der Karte anzeigen:} Durch die identische Vorgehensweise wie bei den Wetterstationen ist es aus denselben Gründen nicht möglich, die \ac{LCZ} auf der Karte anzuzeigen.
    \item \textbf{Login und Registrierung:} Die Login- und Registrierungsfunktionen sind grundlegend durch die Existenz von entsprechenden Formularen umgesetzt. Allerdings findet keine Verknüpfung mit der Datenbank statt, sodass die Nutzer*innen sich nicht einloggen können.
    \item \textbf{Manipulieren der Anzeige der Sensordaten:} Im Bereich \enquote{Sensorinspektor} ist es möglich, einen Zeitraum auszuwählen, um die Sensordaten im Graphen zu manipulieren. Allerdings werden die Sensordaten nicht passend manipuliert.
\end{itemize}

Die folgenden Anforderungen wurden \textbf{nicht} umgesetzt:

\begin{itemize}
    \item \textbf{Einbinden der \ac{LCZ} in den Bereich \enquote{Sensorinspektor}:} Zum Vergleichen der Daten aus dem Bereich \enquote{Sensorinspektor} ist es zusätzlich hilfreich, die \ac{LCZ} Bambergs bei der Analyse berücksichtigen zu können.
    \item \textbf{Bereich \enquote{Mein Bereich} umsetzen:} Nach erfolgter Registrierung und Login soll es den Nutzer*innen möglich sein, einen eigenen Bereich einzusehen, in dem die favorisierten Sensoren, beigefügte Chats und Benachrichtigungen angezeigt werden.
    \item \textbf{Aktuelle Auslesungen der Wetterstationen anzeigen:} Die aktuellen Auslesungen der Wetterstationen sollen in der Karte angezeigt werden. Hierfür ist es notwendig, die Daten der Wetterstationen in Echtzeit zu erhalten.
    \item \textbf{Request-Parametrisierung der URL:} Die URL soll die Parameter der Anfrage\sidenote{z.B. \url{http://bamberg-messen.com/station?id=456&timefrom=2023-03-01&timeto=2023-08-01}, um alle Daten der Station mit der ID 456 im Zeitraum 01.03.2023 bis 01.08.2023 zu erhalten} beinhalten, sodass diese geteilt werden kann. Dadurch können zwischen Nutzer*innen die gleichen Anfragen geteilt werden, um die gleichen Ergebnisse zu erhalten, ohne die Anfrage erneut stellen zu müssen.
    \item \textbf{Barrierefreiheit berücksichtigen:} Die Barrierefreiheit der Applikation soll berücksichtigt werden, damit diese von allen Nutzer*innen verwendet werden kann.
    \item \textbf{Button zum Ein-/Ausblenden von Stationen:} Mithilfe eines Buttons soll es möglich sein, alle oder einzelne Stationen auf der Karte ein- und auszublenden.
    \item \textbf{Vordefinierte Standardsichten:} Diverse Buttons mit Standardsichten (z.B. Sommertage, Tropennächte etc.) sollen in der Karte eingebunden werden, um die Sichtbarkeit der Daten zu erhöhen.
    \item \textbf{Favorisieren von Stationen:} Zum besseren Wiederauffinden von interessanten Wetterstationen sollen diese favorisiert werden können.
    \item \textbf{Benachrichtigungen:} Die Nutzer*innen sollen über neue Daten (Auslesungen, neue Chats, neue Ereignisse etc.) der favorisierten Stationen benachrichtigt werden.
    \item \textbf{Qualitätskontrolle der Sensordaten:} Zum Zeitpunkt dieser Arbeit werden die Sensordaten manuell von einer parallel laufenden Abschlussarbeit eingeholt, welche bereits diversen Qualitätskontrollen unterzogen sind. Diese Qualitätskontrollen werden in der Applikation selbst nicht umgesetzt. Dieser Aspekt soll automatisiert werden. 
\end{itemize}

Eine genaue Dokumentation der Anforderungen ist auf dem GitHub-Repository\sidenote{\url{https://github.com/SamAkc/MasterThesis}} dieses Projekts zu finden. Die Anforderungen sind dabei in Form von Tasks, die sich in einem Kanban-Board\sidenote{Tool für agiles Projektmanagement zum Organisieren von Aufgaben \cite{Rehkopf2023}} befinden, umgesetzt worden.

\section{Evaluation des Projekts}
\label{sec:personal_evaluation}
Das Projekt kann im Allgemeinen als sehr positiv in Bezug auf die Planung, Durchführung und Evaluierung, aber auch in Bezug auf die Kommunikation mit Dritten in Form von Stakeholdern und Ansprechpartner*innen zum Einholen von Unterstützung und externen Daten(-quellen) beschrieben werden. \\ In der Planungsphase wurde durch den großen Informationsfluss eine große Menge an Informationen erhoben, welche zunächst kategorisiert und in kleinere Teilprozesse unterteilt werden musste. Die Durchführungsphase konnte infolgedessen aber schneller und einfacher Fortschritte erzielen. Wenn Probleme oder Hürden aufgetreten sind, ist durch die Hilfe von Ansprechpartner*innen in der Regel stets eine Lösung gefunden worden. \\ Die zeitliche Eingrenzung des Projekts hingegen hatte einen Einfluss auf die Evaluierung: Da die Anforderungen nicht vollständig umgesetzt werden konnten, ist der Aspekt der Evaluierung des interaktiven Tools selbst durch die Stakeholder umso wichtiger, weil auf diese Weise die Effektivität des Einsatzes von Crowdsensing in der Softwareentwicklung (z.B. durch Feedback der Stakeholder) aktiv gemessen werden kann.

Der Aspekt des Crowdsensings in der Softwareentwicklung hat auf Basis dieses Projekts und dem Einsatz der ersten Prototypen einen sinnvollen Ansatz dargestellt: Das Berücksichtigen von Funktionalitäten, die es den Nutzer*innen erlauben, Eingaben an den vorhandenen Daten vorzunehmen, ist in der Implementierung zwar eine sinnvolle, aber komplex umzusetzende Lösung  (vgl. Anforderungsanalyse in Kapitel \ref{sec:requirements} und Architektur in Kapitel \ref{sec:implementation}). Durch einfache Eingabemethoden, die an spezifische Datensätze angehängt werden (und idealerweise mit einer Datenbank verknüpft sind), kann Crowdsensing in Applikationen eingebaut werden.

Grundsätzlich kann die Entwicklung eines Projekts wie dieses als eine sehr wertvolle Erfahrung gesehen werden: Der Planungsprozess hat vermittelt, inwieweit vorab Schritte eingeleitet werden müssen, um potenziell auftretende Probleme und Fragen während der Durchführung zu vermeiden. Zusätzlich hat dieser Prozess miteinbezogen, notwendige Komponenten, die zum Einsatz kommen sollen, zu identifizieren, was eine gründliche Recherche der Anforderungen und der Komponenten selbst vorausgesetzt hat. \\ Die Durchführung des Projekts selbst ist durch die Planung und die Identifizierung der notwendigen Komponenten einfacher gestaltet gewesen, da die notwendigen Schritte bereits vorab festgelegt worden sind. Allerdings ist in der Durchführung die stetige Überprüfung und Anpassung notwendig gewesen, um die Umsetzung der Anforderungen jederzeit gewährleisten zu können. Der hohe Zeitaufwand und die zeitliche Limitierung sind dabei die grundsätzlichen Herausforderungen, die es zu bewältigen gegolten hat. Da das Projekt ein hohes (technisches) Vorwissen durch die Verwendung von vielfältigen Softwarekomponenten voraussetzt, ist es notwendig gewesen, sich in diese Technologien einzuarbeiten, was ebenfalls Zeit in Anspruch genommen hat. 

\section{Evaluation der Zusammenarbeit mit den Stakeholdern}
\label{sec:evaluationstakeholder}
Die Zusammenarbeit mit den Stakeholdern hat sich als durchweg positiv erwiesen. Die Kommunikation ist durch die regelmäßigen Meetings, ihre Erreichbarkeit und die Möglichkeit, Fragen stellen zu können, ohne Einschränkungen möglich gewesen. 

Der \ac{BVM} hat stets angeboten, Fragen zum Klimamessnetz und zu der Nutzung und ihren Erfahrungen mit den Netatmo-Wetterstationen zu beantworten. Mit der Nutzung der Wetterstationen haben die Nutzer*innen ihre Zufriedenheit mitgeteilt und haben keine Fehler oder Anomalien beobachten können. Zusätzlich vertritt der Verein die Meinung, dass der Einsatz eines interaktiven Tools mit einem Crowdsensing-Ansatz ihrer Ansicht nach sinnvoll ist und für ihre Zwecke einen Mehrwert darstellt. 

Der Austausch mit dem \ac{MOBI}-Lehrstuhl hat sich ebenfalls als positiv erwiesen: Hier konnten Fragen zur technischen Umsetzung geklärt werden. Bei Herausforderungen bezüglich der Implementierung in der Durchführungsphase wurden Kontaktpersonen zur Unterstützung vermittelt. Der \ac{MOBI}-Lehrstuhl ist hauptverantwortlich gewesen, die technischen Anforderungen an das Projekt zu stellen.

Die Kommunikation mit dem Domänenexperten Prof.\ Dr.\ Thomas Foken ist auch als durchweg positiv hervorzuheben: Durch das große Domänenwissen des Experten kann Literatur zu der (Mikro-)Meteorologie gefunden, meteorologische Fragen geklärt und Anforderungen dieser Domäne an das Projekt gestellt werden.

\section{Evaluation auf Grundlage der Literatur}
\label{sec:evaluationliteratur}
Um den Vergleich zu ziehen, inwieweit die Vorgehensweise und der aufgestellte Prototyp mit der Forschung übereinstimmt, werden in diesem Kapitel die genannten Aspekte mit der Literatur und damit den verwandten Arbeiten aus Kapitel \ref{sec:related_work} verglichen.

Wie in Kapitel \ref{sec:crowdsensing} bereits definiert, spielen die in der Literatur definierten Charakteristika für eine Crowdsensing-Applikation eine große Rolle. Das mobile Crowdsensing überträgt die (Unter-)Aufgaben an eine (Menschen-)Menge, um eine Lösung für ein gegebenes Problem zu erreichen. Im Falle dieses Prototypen werden die Nutzer*innen dazu angeregt, durch Annotationen in Form von Eingaben an den einzelnen Sensordaten der Stationen, Anomalien und Fehler der Auslesungen zu erkennen, um die Qualität der erhobenen Daten zu verbessern. Des weiteren handelt es sich bei den Daten um eine standort-verteilte Menge, was sich durch die unterschiedlichen Standorte der Stationen begründen lässt. Dabei kommt die menschliche Logik der Nutzer*innen zum Einsatz, die auf Basis von Wissen und Erfahrungen\sidenote{Konkretes Beispiel: Anwohner der Langen Straße in Bamberg, welcher täglich beobachten kann, dass zu einer bestimmten Tageszeit Wärmestrahler der Brauerei Sternla eingeschaltet werden und die Wetterstation sich in unmittelbarer Nähe befindet} an der Verbesserung der Daten beteiligt sind. Da diese Vorgehensweise aufgrund der ehrenamtlichen Natur keine finanziellen Kosten mit sich bringt, handelt es sich um eine kostenlose Lösung.

Die aufgestellten Anforderungen stimmen vollständig mit den verwandten Arbeiten überein. Wenn Daten mit einem Standortbezug vorliegen, kann in allen Fällen in der Literatur in der fertigen Applikation eine interaktive Karte bedient werden, die in Echtzeit die erfassten Daten anzeigt (vgl. \textit{Bodensee Online} und \textit{Budburst}). Inwieweit in der Literatur Anforderungen aufgestellt werden, die Analysen der erfassten Daten ermöglichen sollen (z.B. durch Graphen, Vergleiche untereinander etc.), ist abhängig vom Zweck der jeweiligen Applikation: Für die beiden Beispiele aus der Literatur \textit{Bodensee Online} und \textit{Budburst} kann observiert werden, dass in der Tat die Auslesungen der Sensoren für Analysen verwendet werden. So werden für \textit{Bodensee Online} beim Klick auf die interaktive Karte für den jeweiligen Standpunkt Graphen eingeblendet, die den Verlauf der Wassertemperatur in °C im Verhältnis zu der Strömung in m/s darstellen\sidenote{\url{https://www.lubw.baden-wuerttemberg.de/wasser/bodenseeonline}}. Die Anzeige kann nach eigenem Belieben durch Filterung nach Zeit und Wassertiefe manipuliert werden. \\ Die Applikation \textit{Budburst} hingegen zeigt den Verlauf der gesammelten Daten für einen Standort an: Abhängig davon, an welchem Standort wie viele Einträge zu einer bestimmten Pflanze vorliegen, kann durch Klick auf einen einzelnen Punkt auf der Karte der Verlauf und die Eingaben von anderen Nutzer*innen für diesen eingesehen werden. Dadurch kann durch die Nutzer*innen dann eingesehen werden, inwiefern sich die beobachtete Pflanze über die Zeit verändert hat (hier: in Form von Statusdiagrammen). Beide Applikationen ermöglichen durch die Einblendung von Graphen und Diagrammen ebenfalls eine Analyse der erfassten Daten. 

Zusammengefasst kann geschlussfolgert werden, dass der aufgestellte Prototyp den Anforderungen und Merkmalen der Literatur entspricht und identische Funktionalitäten mit verwandten Softwarelösungen, welche das gleiche Ziel verfolgen, aufweist. Die Anforderungen der Stakeholder und die Anforderungen der Literatur stimmen somit überein.

\section{Limitationen der Arbeit}
\label{sec:limitations}
Als größte Limitation ist der zeitliche Aspekt der Arbeit zu erwähnen: Diese ist einer Bearbeitungsdauer von sechs Monaten unterlegen und hat aufgrund dessen beeinflusst, ob und auf welche Art und Weise Aspekte des Projektes umgesetzt werden konnten. Für geringer priorisierte Komponenten, in Form von Planungsschritten, Umfang der Ausarbeitung, aber auch Anforderungen musste abgewogen werden, inwieweit eine Umsetzung erfolgen konnte. Für nicht-umgesetzte Anforderungen hat gegolten, dass diese in zukünftige Arbeiten ausgelagert werden mussten. Diese Limitation hat zusätzlich bestimmt, inwiefern Prozesse der Arbeit als Prototypen oder vollständig ausgearbeitet werden konnten (vgl. teilweise bzw. nicht-umgesetzte Anforderungen). Unter anderem findet sich dieser Punkt in der Architektur des Tools, weswegen eine Datenbankinstanz als Container, welcher lokal auf der Maschine des Nutzer*innen ausgeführt wird, initiiert wird, statt auf einem global erreichbaren Server. Ähnlich verhält es sich mit der Flask-Instanz: Die Applikation wird lokal auf einem Webserver des Nutzer*innen ausgeführt, statt auf einem Server, sodass eine Verbindung nur mit ausgeführtem Container und lediglich lokal hergestellt werden kann.

In der wissenschaftlichen Ausarbeitung ist der Crowdsensing-Aspekt ebenfalls nur limitiert repräsentativ: Diese Arbeit spiegelt ein Beispiel der Entwicklung eines interaktiven Tools mit einem Crowdsensing-Ansatz wider und kann aufgrund der auftretenden Limitierungen nicht als vollständig repräsentativ angesehen werden, da diverse Komponenten nur konzeptuell oder rudimentär umgesetzt sind. Andere Projekte, welche diesen Ansatz aufgreifen möchten, können ähnliche, aber auch unterschiedliche Ergebnisse und Limitierungen aufweisen, sodass diese Arbeit stattdessen als Grundlage für weitere Forschung angesehen werden soll. Der citizen-science-Ansatz in der Softwareentwicklung ist ein neuer Ansatz, welcher in der Literatur noch nicht ausreichend behandelt wird. Mit dieser Arbeit soll daher die zusätzliche Forschung in dieser Domäne angeregt werden. \\ Beispielhaft können hier andere Daten in die Implementierung aufgenommen werden, um den Sinn und die Relevanz von Crowdsensing in anderen Bereichen zu erforschen (vgl. Kapitel \ref{sec:related_work}).

Des weiteren ist dieses Projekt auf die identifizierten Stakeholder limitiert. Dementsprechend ist das Konzept, die aufgestellten Anforderungen und die Implementierung auf die Bedürfnisse der Stakeholder zugeschnitten. Andere Stakeholder können andere Anforderungen und Bedürfnisse haben, sodass die genannten Komponenten unterschiedlich ausfallen können. 

Zusätzlich ist die Wahl der Softwarekomponenten aufgrund der Recherche und der Erfahrungen des Autors mit den Komponenten und den aktuellen Trends in der Softwareentwicklung getroffen worden. Andere Softwarekomponenten können andere Funktionalitäten, aber auch Limitationen aufweisen, sodass die Wahl der Softwarekomponenten ebenfalls einen Einfluss auf die Ergebnisse der Arbeit hat.

Als letzte Limitation ist die Evaluierung durch die Stakeholder zu erwähnen: Für das Projekt ist es sinnvoll, dass das implementierte Werkzeug von den Stakeholdern getestet wird und Feedback zur Nutzungsweise, aber auch zur Effektivität des Crowdsensing-Ansatzes gegeben wird. Hier ist besonders relevant, inwieweit dieser Ansatz die Nutzungsweise der Nutzer*innen positiv beeinflusst. Durch die zeitliche Limitierung ist dieser Teil der Arbeit nicht umgesetzt, sodass die Evaluierung durch die Stakeholder in zukünftigen Arbeiten oder bei einer Weiterführung dieser Arbeit stattfinden soll. Hierbei ist es sinnvoll, den Vergleich zwischen den bisherigen Analysen der Auslesungen der Wetterstationen des \ac{BVM} mit den Analysen durch den Einsatz des Werkzeugs zu ziehen, um die Effektivität des Crowdsensing-Ansatzes zu evaluieren.

% TODO: würde mir mehr Limitationen überlegen, deine einzige Limitation ist fast nur “ wenig Zeit mimimi” 
% einfach auch unabhängig davon mal auf Nebenaspekte gehen a la Daten, Zeitraum über mehrere 
% Jahren messen, gute Testphasen über einige Zeit usw 
% Einfach da echt bisschen was noch hinhauen wenn du es zeitlich schaffst