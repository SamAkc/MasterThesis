\chapter{Diskussion und Evaluation}
Der Hintergrund dieser Arbeit ist durch die Einführung des Crowd\-sen\-sing-Aspekts in die Softwareentwicklung und dessen Auswirkungen auf die Planung, der Software selbst und die Stakeholder gegeben. In diesem Kapitel werden die Ergebnisse dieser Arbeit diskutiert und evaluiert. Dabei wird zunächst darauf eingegangen, inwiefern die Anforderungen aus Kapitel \ref{sec:requirements} erfüllt sind. \\ Anschließend findet eine Evaluation des allgemeinen Projektes statt, in der zuerst die persönlichen Erfahrungen des Autors mit dem Projekt selbst, den Stakeholdern und anschließend im Vergleich mit der Literatur diskutiert werden. Abschließend werden die Limitationen der Arbeit aufgezeigt.
\label{sec:discussion}

\section{Erfüllen der Anforderungen}
\label{sec:requirements_evaluation}
Da dieses Projekt eine zeitliche Limitierung aufweist, ist es grundsätzlich nicht möglich, alle aufgestellten Anforderungen zu erfüllen. Aufgrund dieser Tatsache spielt die Priorisierung eine große Rolle: Diese ermöglicht, eine Reihenfolge der zu erfüllenden Anforderungen einzuhalten und somit die wichtigsten Anforderungen zuerst zu erfüllen. \\ In die Priorisierung fließt gleichzeitig mit ein, den Crowdsensing-Aspekt umzusetzen, sodass die aufgestellten Leitfragen beantwortet werden können. Konkret bedeutet das, dass folgende Anforderungen \textbf{vollständig} umgesetzt sind:

\begin{itemize}
    \item \textbf{Einsehen einer interaktiven Karte:} Eine OpenStreetMap-Karte ist in der Startseite der Applikation eingebettet.
    \item \textbf{Sensordaten als Graphen im Bereich \enquote{Sensorinspektor} anzeigen:} Mithilfe von iFrames über Grafana werden die Sensordaten als Graphen in der Applikation angezeigt.
    \item \textbf{Qualitätskontrolle der Sensordaten:} Zum Zeitpunkt dieser Arbeit werden die Sensordaten manuell von einer parallel laufenden Abschlussarbeit eingeholt, welche bereits diversen Qualitätskontrollen unterzogen sind. Diese Qualitätskontrollen werden in der Applikation selbst nicht umgesetzt.
\end{itemize}

Die folgenden Anforderungen sind \textbf{teilweise} umgesetzt:

\begin{itemize}
    \item \textbf{Einholen der Sensordaten:} Die Sensordaten werden über eine parallel laufende Abschlussarbeit manuell mithilfe von exportierten CSV-Dateien eingeholt. Dieser Schritt sollte automatisiert werden, um die Daten in Echtzeit zu erhalten.
    \item \textbf{Aufbauen einer Datenbank:} Eine grundlegend funktionierende Datenbank ist mithilfe von Docker umgesetzt und kann in die Applikation eingebunden werden. Diese Datenbank ist jedoch rudimentär und befindet sich lokal in einem Container auf der Maschine des jeweils Nutzenden. Ein Hosting auf einer Cloud oder einem globalen Server sollte angestrebt werden.
    \item \textbf{Chats für den Bereich \enquote{Sensorinspektor}:} Grundlegend ist es möglich, Annotationen an den Graphen im Bereich \enquote{Sensorinspektor} hinzuzufügen. Hierbei wird zur jeweiligen Position am Graphen zusammen mit dem Zeitstempel eine Texteingabe eingefügt. Allerdings findet eine Speicherung der Eingabe in der Datenbank nicht statt, sodass beim Aktualisieren der Applikation die Annotationen verloren gehen. Eine Anbindung an die Datenbank sollte angestrebt werden.
    \item \textbf{Chats für den Bereich \enquote{Meine Karte}:} Es befindet sich ein Chat-Fenster neben der Karte im Bereich \enquote{Meine Karte}. Hier können Nutzende Chats einfügen, die im Anschluss im Chatfenster angezeigt werden. Allerdings werden diese Chats weder in einer Datenbank gespeichert, sodass bei einer Aktualisierung der Seite alle Chats verloren gehen, noch werden diese mit Daten verknüpft.
    \item \textbf{Sensoren auf der Karte anzeigen:} Die Sensoren sind mithilfe von Google Earth Pro auf der Karte markiert. Der nächste Schritt ist das Exportieren von KML-Dateien aus Google Earth Pro, um diese mithilfe von OpenLayers in die OpenStreetMap-Karte einzubinden\sidenote{\url{https://openlayers.org/en/latest/examples/kml.html}}. Allerdings ist dieser Aspekt aus zeitlicher Sicht nicht mehr umsetzbar, sodass die Sensoren nicht auf der Karte angezeigt werden.
    \item \textbf{\ac{LCZ} auf der Karte anzeigen:} Durch die identische Vorgehensweise wie bei den Sensoren ist es aus denselben Gründen nicht möglich, die \ac{LCZ} auf der Karte anzuzeigen.
    \item \textbf{Login und Registrierung:} Die Login- und Registrierungsfunktionen sind grundlegend durch die Existenz von entsprechenden Formularen umgesetzt. Allerdings findet keine Verknüpfung mit der Datenbank statt, sodass die Nutzenden sich nicht einloggen können.
    \item \textbf{Manipulieren der Anzeige der Sensordaten:} Im Bereich \enquote{Sensorinspektor} ist es möglich, einen Zeitraum auszuwählen, um die Sensordaten im Graphen zu manipulieren. Allerdings werden die Sensordaten nicht passend manipuliert.
\end{itemize}

Die folgenden Anforderungen sind \textbf{nicht} umgesetzt:

\begin{itemize}
    \item \textbf{Einbinden der \ac{LCZ} in den Bereich \enquote{Sensorinspektor}:} Zum Vergleichen der Daten aus dem Bereich \enquote{Sensorinspektor} ist es zusätzlich hilfreich, die \ac{LCZ} Bambergs bei der Analyse berücksichtigen zu können.
    \item \textbf{Bereich \enquote{Mein Bereich} umsetzen:} Nach erfolgter Registrierung und Login soll es den Nutzenden möglich sein, einen eigenen Bereich einzusehen, in dem die favorisierten Sensoren, beigefügte Chats und Benachrichtigungen angezeigt werden.
    \item \textbf{Aktuelle Auslesungen der Wetterstationen anzeigen:} Die aktuellen Auslesungen der Wetterstationen sollen in der Karte angezeigt werden. Hierfür ist es notwendig, die Daten der Wetterstationen in Echtzeit zu erhalten.
    \item \textbf{Request-Parametrisierung der URL:} Die URL soll die Parameter der Anfrage\sidenote{z.B. \url{http://bamberg-messen.com/station?id=456&timefrom=2023-03-01&timeto=2023-08-01}, um alle Daten der Station mit der ID 456 im Zeitraum 01.03.2023 bis 01.08.2023 zu erhalten} beinhalten, sodass diese geteilt werden kann. Dadurch können zwischen Nutzenden die gleichen Anfragen geteilt werden, um die gleichen Ergebnisse zu erhalten, ohne die Anfrage erneut stellen zu müssen.
    \item \textbf{Barrierefreiheit berücksichtigen:} Die Barrierefreiheit der Applikation soll berücksichtigt werden, damit diese von allen Nutzenden verwendet werden kann.
    \item \textbf{Button zum Ein-/Ausblenden von Stationen:} Mithilfe eines Buttons soll es möglich sein, alle oder einzelne Stationen auf der Karte ein- und auszublenden.
    \item \textbf{Vordefinierte Standardsichten:} Diverse Buttons mit Standardsichten (z.B. Sommertage, Tropennächte etc.) sollen in der Karte eingebunden werden, um die Sichtbarkeit der Daten zu erhöhen.
    \item \textbf{Favorisieren von Stationen:} Zum besseren Wiederauffinden von interessanten Wetterstationen sollen diese favorisiert werden können.
    \item \textbf{Benachrichtigungen:} Die Nutzenden sollen über neue Daten (Auslesungen, neue Chats, neue Ereignisse etc.) der favorisierten Stationen benachrichtigt werden.
\end{itemize}

Eine genaue Dokumentation der Anforderungen ist auf dem GitHub-Repository\sidenote{\url{https://github.com/SamAkc/MasterThesis}} dieses Projekts zu finden. Die Anforderungen sind dabei in Form von Tasks, die sich in einem Kanban-Board befinden, umgesetzt.

\section{Evaluation des Projekts}
\label{sec:personal_evaluation}
Das Projekt kann im Allgemeinen als sehr positiv in Bezug auf die Planung, Durchführung und Evaluierung, aber auch in Bezug auf die Kommunikation mit Dritten in Form von Stakeholdern und Ansprechpartner*innen zum Einholen von Unterstützung und externe Daten(-quellen) beschrieben werden. \\ In der Planungsphase ist durch das Einholen von Informationen (z.B. durch die Stakeholder, aber auch durch verwandte Arbeiten in der Literatur) eine große Datenmenge gefolgt, welche zunächst kategorisiert und in kleinere Teilprozesse unterteilt werden muss. Zur Folge dieses Schrittes ist es aber möglich, in der Durchführungsphase schneller und einfacher Fortschritte zu erzielen. Wenn Probleme oder Hürden auftreten, ist durch die Hilfe von Ansprechpartner*innen in der Regel stets eine Lösung gefunden. \\ Die zeitliche Eingrenzung des Projekts hingegen hat einen Einfluss auf die Evaluierung: Da die Anforderungen nicht vollständig umgesetzt werden können, ist der Aspekt der Evaluierung des interaktiven Tools selbst durch die Stakeholder umso wichtiger, weil auf diese Weise die Effektivität des Einsatzes von Crowdsensing in der Softwareentwicklung (z.B. durch Feedback der Stakeholder) aktiv gemessen werden kann.

Der Aspekt des Crowdsensings in der Softwareentwicklung stellt auf Basis dieses Projekts und dem Einsatz der ersten Prototypen einen sinnvollen Ansatz dar: Das Berücksichtigen von Funktionalitäten, die es den Nutzenden erlauben, Eingaben an den vorhandenen Daten vorzunehmen, stellt in der Implementierung keine Schwierigkeiten dar. Durch einfache Eingabemethoden, die an spezifische Datensätze angehängt werden (und idealerweise mit einer Datenbank verknüpft sind), kann Crowdsensing auf eine einfache Art und Weise in Applikationen eingebaut werden.

Grundsätzlich kann die Entwicklung eines Projekts wie dieses als eine sehr wertvolle Erfahrung gesehen werden: Der Planungsprozess vermittelt, inwieweit vorab Planungsschritte eingeleitet werden müssen, um potenziell auftretende Probleme und Fragen während der Durchführung zu vermeiden. Zusätzlich schließt dieser Prozess mit ein, notwendige Komponenten, die zum Einsatz kommen sollen, zu identifizieren, was eine gründliche Recherche der Anforderungen und der Komponenten selbst voraussetzt. \\ Die Durchführung des Projekts selbst ist durch die Planung und die Identifizierung der notwendigen Komponenten einfacher gestaltet, da die notwendigen Schritte bereits vorab festgelegt wurden. Allerdings ist es in der Durchführung notwendig, die Planung stets zu überprüfen und gegebenenfalls anzupassen, um die Umsetzung der Anforderungen jederzeit zu gewährleisten. Der hohe Zeitaufwand, die zeitliche Limitierung und das (technische) Vorwissen, welches benötigt wird, sind dabei die grundsätzlichen Herausforderungen, die es zu bewältigen gilt.

\section{Evaluation auf Grundlage der Stakeholder}
\label{sec:evaluationstakeholder}
Die Zusammenarbeit mit den Stakeholdern hat sich als durchweg positiv erwiesen. Die Kommunikation mit den Stakeholdern ist durch die regelmäßigen Meetings, ihrer regelmäßigen Erreichbarkeit und der Möglichkeit, jederzeit Fragen stellen zu können, ohne Einschränkungen möglich. \\ Der \ac{BVM} bietet sich stets an, Fragen zum Klimamessnetz und zu der Nutzung und ihren Erfahrungen mit den Netatmo-Wetterstationen zu beantworten. Mit der Nutzung der Wetterstationen teilen die Nutzenden mit, zufrieden zu sein und keine Fehler oder Anomalien beobachtet zu haben. Zusätzlich vertritt der Verein die Meinung, dass der Einsatz eines interaktiven Tools mit einem Crowdsensing-Ansatz ihrer Ansicht nach sinnvoll ist und für ihre Zwecke einen Mehrwert darstellt. Vom \ac{BVM} werden die Anforderungen zum Erfüllen der Ziele\sidenote{Unter anderem das Erreichen von Verantwortlichen (Politiker*innen, Ämter etc.) zur Reduzierung der Versiegelung, Maximierung von Grünflächen und Verbot von Autos in der Innenstadt} des interaktiven Tools eingeholt.

Der Austausch mit dem \ac{MOBI}-Lehrstuhl hat sich ebenfalls als positiv erwiesen: Hier können Fragen zur technischen Umsetzung geklärt werden. Bei Herausforderungen bezüglich des Erfüllens der Implementierung in der Durchführungsphase werden Kontaktpersonen zur Unterstützung vermittelt. Der \ac{MOBI}-Lehrstuhl ist hauptverantwortlich, die technischen Anforderungen an das Projekt zu stellen.

Die Kommunikation mit dem Domänenexperten Prof.\ Dr.\ Thomas Foken ist ebenfalls als durchweg positiv hervorzuheben: Durch das große Domänenwissen des Experten ist es möglich, Literatur zu der (Mikro-)Meteorologie zu finden, meteorologische Fragen zu klären und Anforderungen dieser Domäne an das Projekt zu stellen.

\section{Evaluation auf Grundlage der Literatur}
\label{sec:evaluationliteratur}

\section{Limitationen der Arbeit}
\label{sec:limitations}
Als größte Limitation ist der zeitliche Aspekt dieser Arbeit zu erwähnen: Diese unterliegt einer Bearbeitungsdauer von sechs Monaten und beeinflusst aufgrund dessen stark, ob und auf welche Art und Weise Aspekte dieses Projektes umgesetzt werden. Für geringer priorisierte Komponenten, in Form von Planungsschritten, Umfang der Ausarbeitung, aber auch Anforderungen muss abgewogen werden, inwieweit eine Umsetzung erfolgt. Für nicht-umgesetzte Anforderungen gilt, dass eine Auslagerung in ein zukünftiges Wiederaufgreifen und Weiterarbeiten stattfindet. Diese Limitation bestimmt zusätzlich, inwiefern Prozesse der Arbeit als Prototypen oder vollständig ausgearbeitet (vgl. teilweise bzw. nicht-umgesetzte Anforderungen) umgesetzt werden. Unter anderem findet sich dieser Punkt in der Architektur des Tools, weswegen eine Datenbankinstanz als Container, welcher lokal auf der Maschine des Nutzenden ausgeführt wird, statt auf einem global erreichbaren Server. Ähnlich verhält es sich mit der Flask-Instanz: Die Applikation wird lokal auf einem Webserver des Nutzenden ausgeführt, statt auf einem Server, sodass eine Verbindung nur mit ausgeführtem Container und lediglich lokal hergestellt werden kann.

In der wissenschaftlichen Ausarbeitung ist der Crowdsensing-Aspekt ebenfalls nur limitiert repräsentativ: Diese Arbeit spiegelt ein Beispiel der Entwicklung eines interaktiven Tools mit einem Crowdsensing-Ansatz wider und kann aufgrund der auftretenden Limitierungen nicht als vollständig repräsentativ angesehen werden, da diverse Komponenten nur konzeptuell oder rudimentär umgesetzt sind. Andere Projekte, welche diesen Ansatz aufgreifen möchten, können ähnliche, aber auch unterschiedliche Ergebnisse und Limitierungen aufweisen, sodass diese Arbeit stattdessen als Grundlage für weitere Forschung angesehen werden soll. Der citizen-science-Ansatz in der Softwareentwicklung ist ein neuer Ansatz, welcher in der Literatur noch nicht ausreichend behandelt wird. Mit dieser Arbeit soll daher die zusätzliche Forschung in dieser Domäne angeregt werden. \\ Beispielhaft können hier andere Daten in die Implementierung aufgenommen werden, um den Sinn und die Relevanz von Crowdsensing in anderen Bereichen zu erforschen (vgl. Kapitel \ref{sec:related_work}).

Als letzte Limitation ist die Evaluierung durch die Stakeholder zu erwähnen: Für dieses Projekt ist es sinnvoll, dass das implementierte Werkzeug von den Stakeholdern getestet wird und Feedback zur Nutzungsweise, aber auch zur Effektivität des Crowdsensing-Ansatzes gegeben wird. Durch die zeitliche Limitierung ist dieser Teil der Arbeit nicht umgesetzt, sodass die Evaluierung durch die Stakeholder in zukünftigen Arbeiten oder bei einer Weiterführung dieser Arbeit stattfinden soll. Hierbei ist es sinnvoll, den Vergleich zwischen den bisherigen Analysen der Auslesungen der Wetterstationen des \ac{BVM} mit den Analysen durch den Einsatz des Werkzeugs zu ziehen, um die Effektivität des Crowdsensing-Ansatzes zu evaluieren.