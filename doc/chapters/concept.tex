\chapter{Konzept} % Kapitel zum Konzept
In diesem Abschnitt der Arbeit geht es darum, das grundlegende Konzept hinter der Entwicklung eines interaktiven Werkzeuges zur Kuratierung von Umweltdaten aufzubauen und zu erläutern. Dabei wird als erster Aspekt der Grundgedanke dieser Arbeit geschildert, um im Anschluss die Methodologie zu erläutern, welche zum Aufstellen der Anforderungen verwendet wird,
um eine Nachvollziehbarkeit und Wiederholung des Projekts gewährleisten zu können. \newline Im nächsten Schritt ist es dann erforderlich, die notwendigen Komponenten der Software zu erfassen, um eine Übersicht über die benötigten Technologien zu erhalten. Im Anschluss wird dann die allgemeine Vorgehensweise für
die gesamte Arbeit erläutert.

\section{Grundgedanke der Arbeit}
Der Grundgedanke eines interaktiven Werkzeugs zur Kuratierung von Umweltdaten ist es, ein bereits vorhandenes System (vgl. Kapitel \ref{sec:ausgangslage} insofern zu erweitern, dass die Erfahrung und Expertise von Menschen als zusätzlicher Qualitätsfilter für die vorliegenden Daten genutzt werden kann, um Anomalien und Fehler, aber auch Kausalitäten für auftretende Phänomene zu erkennen. Die Kuratierung soll dabei durch die Nutzenden des Werkzeugs auf die erhebten Daten durch das bestehende Klimamessnetz in Bamberg erfolgen. Zusätzlich soll es möglich sein, die Sensordaten visuell darzustellen, um Verläufe und Trends zu erkennen, Vergleiche zu ziehen und Ausblicke auf zukünftige Entwicklungen zu erhalten. Die Frage, inwieweit eine solche Kuratierung umgesetzt werden kann (vgl. Kapitel \ref{sec:umsetzung}), welche Auswirkungen dieser Aspekt auf die Softwareentwicklung hat und ob darauf basierend eine Sinnhaftigkeit gegeben ist (vgl. Kapitel \ref{sec:discussion}), soll in dieser Arbeit beantwortet werden. Weiterhin soll diese Arbeit ein Fundament für die Entwicklung eines Crowdsensing-Werkzeugs darstellen, welches in zukünftigen Arbeiten und Projekten aufgenommen und weiterentwickelt werden kann. 

\section{Methodologie zum Aufstellen der Anforderungen}
Das \ac{IEEE} gliedert die Anforderungsanalyse in die Anforderungserhebung (\textit{requirements elicitation}), die Anforderungsanalyse (\textit{requirements analysis}), die Anforderungsspezifikation (\textit{requirements specification}) und Anforderungsbewertung (\textit{requirements validation}) \cite{ieee2004}. An dieser Vorgehensweise orientiert sich auch die Methodologie, welche in dieser Arbeit verwendet wird.

Zum Erheben der Anforderungen ist es zunächst erforderlich, die Stakeholder zu identifizeren, da diese die Nutzenden der Software sind und somit die Anforderungen stellen. Im Falle dieser Arbeit wurden die Stakeholder bereits vor Beginn des Projekts festgelegt (siehe Kapitel \ref{sec:stakeholder}). Nachdem die Stakeholder identifiziert worden sind, ist es möglich gewesen, diese direkt nach den Anforderungen zu befragen. Vom \ac{BVM}\sidenote{https://www.bvm-bamberg.de/de/projektseiten/klima/} finden in regelmäßigen Abständen Treffen statt, in welchen zusammen mit Domänenexperte Prof. Dr. Thomas Foken Analysen über die aktuellen Auswertungen der Wetterstationen durchgeführt werden. Dabei wird unter anderem auch darüber gesprochen, wie diese Werte zustande kommen, wie sie sich entwickeln könnten und was sie bedeuten. Da an diesen Treffen in der Regel auch die anderen Stakeholder teilgenommen haben, konnten diese dazu genutzt werden, Fragen zu Anforderungen an ein interaktives Werkzeug zur Kuratierung von Umweltdaten zu erheben. Insbesondere ist es wichtig gewesen, das Ziel und den Zweck des \ac{BVM} zu erfragen, da sich aus dieser Frage bereits Anforderungen an das Werkzeug ergeben und wie die Priorisierung bei der darauffolgenden Analyse zu gestalten werden muss. Zusätzlich wurden die Stakeholder dazu befragt, welche Funktionalitäten sie sich für ein solches Tool wünschen und auf welche Art und Weise sie sich vorstellen könnten, ein solches zu verwenden. Die Gespräche wurden protokolliert und im Anschluss in der Analyse (siehe Kapitel \ref{sec:requirements} ausgewertet. Der Schritt der \textit{requirements specification} wird in \ref{sec:requirementsspecification} näher betrachtet. \newline Hier werden die aufgestellten Anforderungen in einer Spezifikation gesammelt, die idealerweise Charakterisitika wie Korrektheit, Vollständigkeit, Eindeutigkeit oder einer Verifizierbarkeit aufweist \cite{institute1984ieee}. Im Anschluss muss eine Bewertung der Anforderungen erfolgen, um sicherzustellen, dass diese auch tatsächlich die Bedürfnisse der Stakeholder erfüllen. Dazu wird ein Prototyp übergeben, mit dessen Hilfe ein erstes Hands-on stattfinden kann. Die Stakeholder werden dann nach ihrem Feedback befragt und die Anforderungen werden entsprechend angepasst. Dieser Schritt wird in Kapitel \ref{sec:evaluationstakeholder} ausführlicher beschrieben.

\section{Erfassung der notwendigen Komponenten der Software}
Dieses Kapitel handelt davon, die erforderlichen Komponenten zu erfassen, um im Anschluss die Architektur in Kapitel \ref{sec:implementation} aufstellen zu können. \newline Aufgrund der Tatsache, dass das Werkzeug im fertigen Zustand eine Webanwendung darstellen soll, welche jederzeit unabhängig vom Standort zugänglich ist, bietet es sich an, die Grundlage der Software mit einer Programmiersprache umzusetzen, die auf eine einfache Art und Weise sowohl eine Webanwendung erstellen, als auch durch das Hinzufügen weiterer Komponenten diese Anwendung erweitern kann. \newline Da grundsätzlich mit Daten gearbeitet werden soll (in Form von Sensordaten, aber auch notwendige Daten zur Funktionsweise einer Applikation, wie z.B. Benutzerdaten) spielt die Verwaltung dieser Daten eine große Rolle. Dieser Aspekt ist insofern wichtig, da die Daten sowohl von der Webanwendung angelegt (z.B. Login-Daten, Feedback der Nutzenden), als auch visualisiert werden müssen (z.B. Historie der Sensordaten einer Wetterstation). Dementsprechend ist eine Konnektivität zwischen diesen beiden Komponenten essenziell für die Funktionsweise der Software. \newline 

\section{Umsetzung}
\label{sec:umsetzung}
Um eine Nachvollziehbarkeit dieser Arbeit gewährleisten zu können, soll in diesem Abschnitt erläutert werden, wie die grundsätzliche Arbeit umgesetzt wurde. \newline
Aufgrund der zeitlichen Limitierung dieser Arbeit ist die Anwendung von Ansätzen aus der agilen Softwareentwicklung notwendig gewesen: Diese beschreiben einen Prozess, der durch die Flexibilität der Software (durch Aufteilen in einfachere Teilprozesse) und den kontinuierlichen Austausch mit den Stakeholdern einen schnelleren Einsatz der Software ermöglicht und die Minimierung von Risiken im Entwicklungsprozess erlaubt \cite{Siepermann2018}. Eine sehr beliebte Methode ist hier die iterative Entwicklung, die sich insbesondere dann anbietet, wenn die Projektziele und der Zeitplan klar definiert sind, die Vorgehensweise, diese zu erreichen, aber noch unklar ist [...]