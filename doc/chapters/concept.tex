\chapter{Konzept} % Kapitel zum Konzept
% Grundgedanke hinter dem Tool erläutern, 
% einen Entwurf darstellen und mithilfe von Use Cases verstärken
In diesem Abschnitt der Arbeit geht es darum, das grundlegende Konzept hinter der Entwicklung eines interaktiven Werkzeuges zur Kuratierung 
von Umweltdaten aufzubauen und zu erläutern. Dabei ist es zunächst erforderlich, die allgemeine Umsetzung von Crowdsensing in einer solchen Software zu analysieren, um 
darauffolgend Parallelen zu \enquote{Bamberg messen 2.0} auffinden zu können. \newline Im nächsten Schritt ist es dann erforderlich, den Entwurf näher zu untersuchen damit 
die Nachvollziehbarkeit der Methodologie der Entwicklung gewährleistet ist. Im Anschluss können somit Use Cases aufgestellt werden, um Parallelen zu alltäglichen Ereignissen 
schlagen zu können, sodass die Verwendbarkeit eines solchen Werkzeuges greifbarer wird.
\section{Grundgedanke hinter einem Crowdsensing-Tool}

\section{Entwurf}

\section{Use Cases}