\chapter{Konzept} % Kapitel zum Konzept
In diesem Abschnitt der Arbeit geht es darum, das grundlegende Konzept hinter der Entwicklung eines interaktiven Werkzeuges zur Kuratierung von Umweltdaten aufzubauen und zu erläutern. Dabei ist es zunächst erforderlich, 
die Vorgehensweise zu erläutern, welche zum Aufstellen der Anforderungen, die für die Entwicklung eines interaktiven Werkzeugs benötigt werden, verwendet wird, um eine Nachvollziehbarkeit und Wiederholung des Projekts gewährleisten zu 
können. \newline Im nächsten Schritt ist es dann erforderlich, die notwendigen Komponenten der Software zu erfassen, um eine Übersicht über die benötigten Technologien zu erhalten. Im Anschluss kann dann die Umsetzung des Werkzeuges auf Grundlage 
der aufgestellten Anforderungen und der notwendigen Komponenten erfolgen.

\section{Methodologie zum Aufstellen der Anforderungen}
Das \ac{IEEE} gliedert die Anforderungsanalyse in die Anforderungserhebung (\textit{requirements elicitation}), die Anforderungsanalyse (\textit{requirements analysis}), die Anforderungsspezifikation (\textit{requirements specification}) und Anforderungsbewertung 
(\textit{requirements validation}) \cite{ieee2004}. An dieser Vorgehensweise orientiert sich auch die Methodologie, welche in dieser Arbeit verwendet wird.

Zum Erheben der Anforderungen ist es zunächst erforderlich, die Stakeholder zu identifizeren, da diese 
die Nutzenden der Software sind und somit die Anforderungen stellen. Im Falle dieser Arbeit wurden die Stakeholder bereits vor Beginn des Projekts festgelegt (siehe Kapitel \ref{sec:stakeholder}). Nachdem die Stakeholder identifiziert worden sind, ist es möglich gewesen, 
diese direkt nach den Anforderungen zu befragen. Vom \ac{BVM}\sidenote{https://www.bvm-bamberg.de/de/projektseiten/klima/} finden in regelmäßigen Abständen Treffen statt, in welchen zusammen mit Domänenexperte Prof. Dr. Thomas Foken Analysen über die aktuellen Auswertungen der Wetterstationen durchgeführt werden. Dabei wird unter anderem 
auch darüber gesprochen, wie diese Werte zustande kommen, wie sie sich entwickeln könnten und was sie bedeuten. Da an diesen Treffen in der Regel auch die anderen Stakeholder teilgenommen haben, konnten diese dazu genutzt werden, Fragen zu Anforderungen an ein interaktives Werkzeug 
zur Kuratierung von Umweltdaten zu erheben. Insbesondere ist es wichtig gewesen, das Ziel und den Zweck des \ac{BVM} zu erfragen, da sich aus dieser Frage bereits Anforderungen an das Werkzeug ergeben und wie die Priorisierung bei der darauffolgenden Analyse zu gestalten werden muss. Zusätzlich wurden die Stakeholder dazu befragt, welche Funktionalitäten 
sie sich für ein solches Tool wünschen und auf welche Art und Weise sie sich vorstellen könnten, ein solches zu verwenden. Die Gespräche wurden protokolliert und im Anschluss in der Analyse (siehe Kapitel \ref{sec:requirements} ausgewertet. Der Schritt der \textit{requirements specification} wird in \ref{sec:requirementsspecification} näher betrachtet. Hier 
werden die aufgestellten Anforderungen in einer Spezifikation gesammelt, die idealerweise Charakterisitika wie Korrektheit, Vollständigkeit, Eindeutigkeit oder einer Verifizierbarkeit aufweist \cite{institute1984ieee}. Im Anschluss muss eine Bewertung der Anforderungen erfolgen, um sicherzustellen, dass diese auch tatsächlich die Bedürfnisse der Stakeholder erfüllen. 
Dazu wird ein Prototyp übergeben, mit dessen Hilfe ein erstes Hands-on stattfinden kann. Die Stakeholder werden dann nach ihrem Feedback befragt und die Anforderungen werden entsprechend angepasst. Dieser Schritt wird in Kapitel \ref{sec:evaluation} ausführlicher beschrieben.

\section{Erfassung der notwendigen Komponenten der Software}
In diesem Kapitel geht es darum, die erforderlichen Komponenten zu erfassen, um im Anschluss die Architektur in Kapitel \ref{sec:implementation} aufstellen zu können. Aufgrund der Tatsache, dass das Werkzeug im fertigen Zustand eine Webanwendung darstellen soll, welche jederzeit unabhängig vom Standort zugänglich ist, bietet es sich an, die Grundlage der Software mit einer Programmiersprache 
umzusetzen, die auf eine einfache Art und Weise sowohl eine Webanwendung erstellen, als auch durch das Hinzufügen weiterer Komponenten diese Anwendung erweitern kann. Ein Beispiel einer zur Erweiterung genutzten Komponente ist das Einbinden einer Karte, auf welchem die Stationen von Netatmo, die vordefinierten lokalen Klimazonen und die Auslesungen der Stationen sichtbar sind.

\section{Umsetzung}
