\chapter{Konzept} % Kapitel zum Konzept
In diesem Abschnitt der Arbeit geht es darum, das grundlegende Konzept hinter der Entwicklung eines interaktiven Werkzeuges zur Kuratierung von Umweltdaten aufzubauen und zu erläutern. Dabei ist es zunächst erforderlich, 
die Vorgehensweise zu erläutern, welche zum Aufstellen der Anforderungen, die für die Entwicklung eines interaktiven Werkzeugs benötigt werden, verwendet wird, um eine Nachvollziehbarkeit und Wiederholung des Projekts gewährleisten zu 
können. \newline Im nächsten Schritt ist es dann erforderlich, die notwendigen Komponenten der Software zu erfassen, um eine Übersicht über die benötigten Technologien zu erhalten. Im Anschluss kann dann die Umsetzung des Werkzeuges auf Grundlage 
der aufgestellten Anforderungen und der notwendigen Komponenten erfolgen.

\section{Methodologie zum Aufstellen der Anforderungen}

\section{Erfassung der notwendigen Komponenten der Software}

\section{Umsetzung}
