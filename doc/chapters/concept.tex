\chapter{Konzept} % Kapitel zum Konzept
In diesem Abschnitt der Arbeit geht es darum, das grundlegende Konzept hinter der Entwicklung eines interaktiven Werkzeuges zur Kuratierung von Umweltdaten aufzubauen und zu erläutern. Dabei wird als erster Aspekt der Grundgedanke dieser Arbeit geschildert, um im Anschluss die Methodologie zu erläutern, welche zum Aufstellen der Anforderungen verwendet wird,
um eine Nachvollziehbarkeit und Wiederholung des Projekts gewährleisten zu können. \\ Im nächsten Schritt ist es dann erforderlich, die notwendigen Komponenten der Software zu erfassen, um eine Übersicht über die benötigten Technologien zu erhalten. Im Anschluss wird dann die allgemeine Vorgehensweise für die gesamte Arbeit erläutert.

\section{Grundgedanke der Arbeit}
Der Grundgedanke eines interaktiven Werkzeugs zur Kuratierung von Umweltdaten ist es, ein bereits vorhandenes System (vgl. Kapitel \ref{sec:ausgangslage}) insofern zu erweitern, dass die Erfahrung und Expertise von Menschen als zusätzlicher Qualitätsfilter für die vorliegenden Daten genutzt werden kann, um Anomalien und Fehler, aber auch Kausalitäten für auftretende Phänomene zu erkennen. Die Kuratierung soll dabei durch die Nutzenden des Werkzeugs auf die erhobenen Daten durch das bestehende Klimamessnetz in Bamberg erfolgen. Zusätzlich soll es möglich sein, die Sensordaten visuell darzustellen, um Verläufe und Trends zu erkennen, Vergleiche zu ziehen und Ausblicke auf zukünftige Entwicklungen zu erhalten. Die Frage, inwieweit eine solche Kuratierung umgesetzt werden kann (vgl. Kapitel \ref{sec:umsetzung}), welche Auswirkungen dieser Aspekt auf die Softwareentwicklung hat und ob darauf basierend eine Sinnhaftigkeit gegeben ist (vgl. Kapitel \ref{sec:discussion}), soll in dieser Arbeit beantwortet werden. Weiterhin soll diese Arbeit ein Fundament für die Entwicklung eines Crowdsensing-Werkzeugs darstellen, welches in zukünftigen Arbeiten und Projekten aufgenommen und weiterentwickelt werden kann. 

\section{Methodologie zum Aufstellen der Anforderungen}
\label{sec:methodologyrequirements}
Das \ac{IEEE} gliedert die Anforderungsanalyse in die Anforderungserhebung (\textit{requirements elicitation}), die Anforderungsanalyse (\textit{requirements analysis}), die Anforderungsspezifikation (\textit{requirements specification}) und Anforderungsbewertung (\textit{requirements validation}) \cite{ieee2004}. An dieser Vorgehensweise orientiert sich auch die Methodologie, welche in dieser Arbeit verwendet wird.

Zum Erheben der Anforderungen ist es zunächst erforderlich, die Stakeholder zu identifizieren, da diese die Nutzenden der Software sind und somit die Anforderungen stellen. Im Falle dieser Arbeit wurden die Stakeholder bereits vor Beginn des Projekts festgelegt (siehe Kapitel \ref{sec:stakeholder}). Nachdem die Stakeholder identifiziert worden sind, ist es möglich gewesen, diese direkt nach den Anforderungen zu befragen. Vom \ac{BVM}\sidenote{\url{https://www.bvm-bamberg.de/de/projektseiten/klima/}} finden in regelmäßigen Abständen Treffen statt, in welchen zusammen mit Domänenexperte Prof.\ Dr.\ Thomas Foken Analysen über die aktuellen Auswertungen der Wetterstationen durchgeführt werden. Dabei wird unter anderem auch darüber gesprochen, wie diese Werte zustande kommen, wie sie sich entwickeln könnten und was sie bedeuten. Da an diesen Treffen in der Regel auch die anderen Stakeholder teilgenommen haben, konnten diese dazu genutzt werden, Fragen zu Anforderungen an ein interaktives Werkzeug zur Kuratierung von Umweltdaten zu erheben. Insbesondere ist es wichtig gewesen, das Ziel und den Zweck des \ac{BVM} zu erfragen, da sich aus dieser Frage bereits Anforderungen an das Werkzeug ergeben und wie die Priorisierung bei der darauffolgenden Analyse zu gestalten werden muss. Zusätzlich wurden die Stakeholder dazu befragt, welche Funktionalitäten sie sich für ein solches Tool wünschen und auf welche Art und Weise sie sich vorstellen könnten, ein solches zu verwenden. Die Gespräche wurden protokolliert und im Anschluss in der Analyse (siehe Kapitel \ref{sec:requirements}) ausgewertet. Der Schritt der \textit{requirements specification} wird in \ref{sec:requirementsspecification} näher betrachtet. \\ Hier werden die aufgestellten Anforderungen in einer Spezifikation gesammelt, die idealerweise Charakteristika wie Korrektheit, Vollständigkeit, Eindeutigkeit oder einer Verifizierbarkeit aufweist \cite{institute1984ieee}. Im Anschluss muss eine Bewertung der Anforderungen erfolgen, um sicherzustellen, dass diese auch tatsächlich die Bedürfnisse der Stakeholder erfüllen. Dazu wird ein Prototyp übergeben, mit dessen Hilfe ein erstes Hands-on stattfinden kann. Die Stakeholder werden dann nach ihrem Feedback befragt und die Anforderungen werden entsprechend angepasst. Dieser Schritt wird in Kapitel \ref{sec:evaluationstakeholder} ausführlicher beschrieben. 

\section{Erfassung der notwendigen Komponenten der Software}
Dieses Unterkapitel handelt davon, die erforderlichen Komponenten zu erfassen, um im Anschluss die Architektur in Kapitel \ref{sec:implementation} aufstellen zu können. \\ Aufgrund der Tatsache, dass das Werkzeug im fertigen Zustand eine Webanwendung darstellen soll, welche jederzeit unabhängig vom Standort zugänglich ist, bietet es sich an, die Grundlage der Software mit einer Programmiersprache umzusetzen, die auf eine einfache Art und Weise sowohl eine Webanwendung erstellen, als auch durch das Hinzufügen weiterer Komponenten diese Anwendung erweitern kann. \\ Da grundsätzlich mit Daten gearbeitet werden soll (in Form von Sensordaten, aber auch notwendige Daten zur Funktionsweise einer Applikation, wie z.B. Benutzerdaten) spielt die Verwaltung dieser Daten eine große Rolle. Dieser Aspekt ist insofern wichtig, da die Daten sowohl von der Webanwendung angelegt (z.B. Login-Daten, Feedback der Nutzenden), als auch visualisiert werden müssen (z.B. Historie der Sensordaten einer Wetterstation). Dementsprechend ist eine Konnektivität zwischen diesen beiden Komponenten essenziell für die Funktionsweise der Software. \\ Zur grundlegenden Verwaltung der Komponenten, der Umgebung, der Daten und der Projektplanung hat sich GitHub\sidenote{\url{https://github.com}} als hilfreich erwiesen: Das Anlegen eines GitHub-Repositories ist kostenlos und kann ohne Einschränkungen geteilt werden. Durch das Anlegen eines Projekts\sidenote{\url{https://github.com/users/SamAkc/projects/2}} können Tickets angelegt werden, die als Aufgaben fungieren und in einem Issue-Board in verschiedene Abschnitte unterteilt werden können, basierend auf dem aktuellen Status des Tickets (hier: \textit{TO-DO}, \textit{In Progress}, \textit{Needs Review}, \textit{Done}, \textit{Future Work}). In den Tickets selbst wird die Problembeschreibung bzw. Aufgabenstellung, die Lösung und Verantwortliche für die Bearbeitung festgehalten. Zusätzlich können durch Kommentare weitere Informationen festgehalten werden, die für die Bearbeitung des Tickets relevant sind (z.B. Bearbeitungsdauer, Lösungen oder Hinweise bei Wechsel der Verantwortlichen). \\ Die Versionsverwaltung mit Git\sidenote{\url{https://git-scm.com/}} bietet dabei die Möglichkeit, geschriebenen Code in einem Repository zu speichern und bei Bedarf auf einen früheren Stand zurückzugreifen. Dies ist insbesondere dann hilfreich, wenn Änderungen am Code vorgenommen werden, die sich als fehlerhaft herausstellen und somit die Funktionsweise der Software beeinträchtigen. Durch die Versionsverwaltung ist es möglich, den Code auf einen früheren Stand zurückzusetzen und die Änderungen zu verwerfen. 

\section{Umsetzung}
\label{sec:umsetzung}
Um eine Nachvollziehbarkeit dieser Arbeit gewährleisten zu können, soll in diesem Abschnitt erläutert werden, wie die grundsätzliche Arbeit umgesetzt wurde. \\
Aufgrund der zeitlichen Limitierung dieser Arbeit ist die Anwendung von Ansätzen aus der agilen Softwareentwicklung notwendig gewesen: Diese beschreiben einen Prozess, der durch die Flexibilität der Software, was durch das Aufteilen der Hauptaufgabe in einfacher zu lösende Teilprozesse erreicht wird und den kontinuierlichen Austausch mit den Stakeholdern einen schnelleren Einsatz der Software ermöglicht und die Minimierung von Risiken im Entwicklungsprozess erlaubt \cite{Siepermann2018}. Eine sehr beliebte Methode ist hier die iterative Entwicklung, die sich insbesondere dann anbietet, wenn die Projektziele und der Zeitplan klar definiert sind, die Vorgehensweise, diese zu erreichen, aber noch unklar ist \cite{salo2007iterative}. Durch den iterativen Ansatz ist es zudem auch möglich, die Software schnell und adaptiv auf die Stakeholder ausrichten zu können, sowohl die höhere Kundenzufriedenheit, als auch eine geringere Defekt-Rate zu erzielen, da eine kontinuierliche face-to-face-Kommunikation zwischen Entwicklung und Stakeholder als primärer Wissensaustausch erfolgt \cite{salo2007iterative}. Der Verbesserungsprozess mithilfe von Iterationen wird in der Literatur in folgende Schritte unterteilt, auf welche in den nachfolgenden Unterkapiteln im Detail eingegangen wird: Vorbereitung, Sammeln von Erfahrungen, Planung von Verbesserungsschritten, Piloting, Follow-up und Validierung und Aufbewahrung \cite{salo2007iterative}.

\subsection{Vorbereitung}
In der Vorbereitungsphase geht es darum, die Frequenz und den Zeitplan der sogenannten \ac{PIW}s\sidenote{Eine \ac{PIW} entspricht einer Review im Team nach jeder erfolgten Iteration um über aufgetretene Hindernisse und Probleme zu diskutieren und Lösungsansätze zu schaffen \cite{salo2007iterative}} festzulegen \cite{salo2007iterative}. Da diese Arbeit von einem einzigen Entwickler für die Stakeholder erfolgt ist, wurden die PIWs im zweiwöchigen Takt durchgeführt. Diese haben sich so gestaltet, dass aufgetretene Probleme auf einem Issue-Board in einem GitHub-Repository\sidenote{\url{https://github.com/SamAkc/MasterThesis}} gesammelt wurden und nach Ende jeder Iteration explizit Lösungen für jene gesucht wurden, statt diese direkt zu beheben. So ist es möglich gewesen, die Arbeitszeit in eine effektive Arbeitsphase und Evaluierungsphase zu unterteilen. Sollte ein Problem nicht gelöst werden können, so wurden Ansprechpartner*innen gesucht, die bei der Lösung behilflich sein konnten --- in der Regel waren dies die Stakeholder.

\subsection{Sammeln von Erfahrungen}
Diese Phase dient dazu, zum ersten PIW die Prozessverbesserungen auf die vom Softwareentwicklungsteam identifizierten Hindernisse und Probleme aufzubauen \cite{salo2007iterative}. Hier werden positive oder negative Erfahrungen, die bei der Softwareentwicklung aufgetreten sind, gesammelt und im Team analysiert, um in späteren Iterationen diese Erfahrungen zu berücksichtigen und einfachere Problemlösungen zu schaffen. Da diese Phase ein Team voraussetzt, konnte dieser Teil der Arbeit nur limitiert durchgeführt werden: Die positiven Erfahrungen in Form von Lösungen für aufgetretene Probleme wurden festgehalten, sodass diese für eventuell auftretende negative Erfahrungen in zukünftigen Iterationen als Lösungsansatz verwendet werden konnten. Negative Erfahrungen hingegen wurden nach erfolgter Identifikation festgehalten, um diese im nächsten Schritt weiter verarbeiten zu können. \\
Für negative Erfahrungen hingegen wurden zusammen mit Ansprechpartnern (in Form von Betreuenden, aber auch Domänenexperten mit Hintergrundwissen, wie z.B. Softwareentwickler in Unternehmen, welche mit identischen/ähnlichen Komponenten arbeiten) Lösungsansätze gesucht, um diese zu lösen oder gar in zukünftigen Iterationen zu vermeiden.

\subsection{Planung von Verbesserungsschritten}
Für die im vorherigen Abschnitt der Arbeit genannten negativen Erfahrungen sollen in diesem Schritt Ansätze zur Verbesserung abgeleitet werden. Dabei werden die Hindernisse und Probleme aus der vorherigen Iteration in einer sogenannten \enquote{Problem Area} gesammelt und aus dem Hintergrundwissen von den Betreuenden dieser Arbeit, aber auch von Domänenexperten (z.B. Softwareentwickler*innen in einem Unternehmen, welches mit identischen/ähnlichen Komponenten arbeitet) nützliche Verbesserungsmöglichkeiten formuliert \cite{salo2007iterative}. 

\subsection{Piloting, Follow-up und Validierung der Verbesserungsschritte}
Die vorher abgeleiteten Ansätze zur Verbesserung werden in der Phase des Pilotings umgesetzt. Neben der Umsetzung ist die Sammlung von Messdaten erforderlich, welche zu Validierungszwecken in den nächsten Phasen verwendet werden \cite{salo2007iterative}. \\ In dieser Arbeit wurde die Umsetzung der Verbesserungsschritte in der Regel in der nächsten Iteration umgesetzt, da die Umsetzung der Verbesserungsschritte in der Regel nicht viel Zeit in Anspruch genommen hat. Zum Zeitpunkt dieser Arbeit ist es nicht möglich gewesen, Messdaten neben der Umsetzung zu sammeln, da diese erst nach der Umsetzung der Verbesserungsschritte generiert werden konnten und zum jetzigen Zeitpunkt noch nicht ausreichend Daten vorhanden sind, um diese für Validierungszwecke zu verwenden. Aus diesem Grund hat auch kein Follow-up der Verbesserungsschritte stattgefunden, da es für diese erforderlich ist, die Messdaten als Grundlage zur Evaluierung der erfolgreichen Anwendung der Verbesserungsschritte zu verwenden \cite{salo2007iterative}. 

\subsection{Aufbewahrung der Verbesserungsschritte}
In der Literatur wird empfohlen, die Ergebnisse der PIWs systematisch in Dokumenten oder Formularen zu sammeln, um diese in zukünftigen Iterationen (für Follow-ups und Validierungen) verwenden zu können \cite{salo2007iterative}. Da diese Vorgehensweise aber den Rahmen dieser Arbeit zeitlich gesprengt hätte, wurden die Ergebnisse der PIWs im GitHub-Repository\sidenote{\url{https://github.com/SamAkc/MasterThesis}} gesammelt. Eine beispielhafte negative Erfahrung in dieser Arbeit hat sich in der Schnittstelle zweier Komponenten gefunden (vgl. Kapitel \ref{sec:implementation}): Diese haben isoliert voneinander ordnungsgemäß funktioniert, sind aber nicht in der Lage gewesen, miteinander zu kommunizieren um den Austausch von Daten zu ermöglichen. Die Lösung für dieses Problem wurde von einem Domänenexperten vorgeschlagen, der bereits Erfahrung mit der Verwendung dieser Komponenten hatte und somit eine funktionierende Lösung für dieses Problem anbieten konnte --- diese Lösung, zusammen mit anderen Lösungen für aufgetretene Probleme und damit, negative Erfahrungen, wurde in einer \enquote{Solution Area} (vgl. Abschnitt \enquote{Troubleshooting} in README-Datei in GitHub-Repository) festgehalten, um diese in zukünftigen Iterationen verwenden zu können. 