\chapter{Fazit und Ausblick}

In dieser Arbeit und der damit verbundenen Softwarelösung wurde sowohl der Prozess der Entwicklung des interaktiven Werkzeuges selbst, als auch die damit verbundenen Auswirkungen sowohl auf die Stakeholder, als auch auf die Softwareentwickler*innen und Daten selbst untersucht. Dabei wurde zuerst der Klimawandel in Deutschland und dessen Auswirkungen näher untersucht, sodass im nächsten Schritt eine Begriffsunterscheidung zum besseren Verständnis dieser Arbeit stattfinden konnte. Darauf basierend wurde ein Konzept vorgestellt, welches die Entwicklung des interaktiven Werkzeuges beschreibt. Um mit der Implementierung beginnen zu können, mussten zunächst Anforderungen aufgestellt und analysiert werden. Anhand dieser Anforderungen wurden User Stories und Use Cases formuliert, welche dann in der eigentlichen Implementierung eingebettet werden mussten. Im Anschluss wurden die aufgestellten Anforderungen und der gesamte Prozess der Entwicklung des interaktiven Werkzeuges evaluiert. Dabei wurde festgestellt, dass die Anforderungen größtenteils erfüllt wurden und ein erster Prototyp erfolgreich entwickelt wurde. Durch die Zuhilfenahme der Literatur konnte der positive Einfluss des Crowdsensings auf die Entwicklung von Software festgestellt werden.

Diese Arbeit dient als Motivation zur Integration des Crowdsensing-Ansatzes in zukünftige Softwarelösungen. Dabei sollte jedoch beachtet werden, dass unterschiedliche Einsatzzwecke der Software unterschiedliche Ergebnisse liefern können, sodass eine allgemeine Aussage über den Einfluss des Crowdsensings auf die Softwareentwicklung nicht getroffen werden kann. Es ist jedoch davon auszugehen, dass die Integration des Crowdsensings in Softwarelösungen, die mit standortbasierten Daten und Eingaben von Nutzer*innen arbeiten, einen positiven Einfluss nachweisen kann. Zusätzlich kann die hier aufgebaute Lösung als erster Prototyp für eine fortgeführte Lösung angesehen werden: Durch die kontinuierliche Wartung, regelmäßigen (Funktions-)Updates, der Integration der Datenquellen in Echtzeit und dem Erfüllen der offenen Anforderungen kann auf Basis dieser Arbeit eine vollwertige Softwarelösung entwickelt werden. Diese kann dann in zukünftigen Projekten\sidenote{wie z.B. vom \ac{BVM}, der Stadt Bamberg, aber auch von anderen Städten} eingesetzt und ausgebaut werden.

Damit stellt sich die Frage, ob und inwieweit die Stadt Bamberg, aber auch Deutschland die aufgestellten Klimaziele erreicht und das Voranschreiten des Klimawandels verlangsamt. Solange kein globales Umdenken stattfindet, wird der Klimawandel weiterhin voranschreiten und die Auswirkungen werden sich weiterhin verschlimmern. Es ist daher wichtig, dass die Stadt Bamberg und Deutschland als Ganzes weiterhin an der Erreichung der Klimaziele arbeiten und diese auch erreichen.