\chapter{Anforderungsanalyse}
\label{sec:requirements}
% Wie ist die Ausgangslage (in Bamberg)? Wie werden/wurden die Anforderungen aufgestellt, 
% wer sind die Stakeholder und warum? Dann Anforderungen definieren und priorisieren
Bevor mit der eigentlichen Implementierung begonnen werden kann, ist es notwendig, die Anforderungsanalyse. Der erste Schritt hierzu besteht daraus, die Ausgangslage in Bamberg zu untersuchen, um darauf aufbauend die Stakeholder zu identifizieren, da diese den Ursprung der Anforderungen darstellen. Nachdem das Aufstellen der Anforderungen erfolgreich gewesen ist, können diese im Anschluss analysiert, also priorisiert, auf ihre Validität und im Anschluss auf ihre Realisierbarkeit überprüft werden. Basierend auf den Anforderungen werden dann Use Cases formuliert, die die Anforderungen in einer konkreten Situation beschreiben und damit die Grundlage für die Implementierung bilden.

\section{Ausgangslage in Bamberg}
\label{sec:ausgangslage}
Bamberg\sidenote{\url{https://www.stadt.bamberg.de/Unsere-Stadt/Stadtinfo/}} ist eine kreisfreie Stadt in Bayern im Regierungsbezirk Oberfranken mit einer Einwohnerzahl von knapp 80.000 Einwohnern (Stand: Dezember 2022). Die Stadt selbst kann dabei in sogenannte \ac{LCZ}, also Klassifikationen, welche sich durch die logische Aufteilung einer Landschaft ergeben \cite{stewart2011local}, unterteilt werden. Zusammen mit dem Domänenwissen des Meteorologen Prof.\ Dr.\ Thomas Foken (vgl. Kapitel \ref{sec:stakeholder}) hat eine rudimentäre Unterteilung in sechs LCZ stattgefunden: Die Innenstadt, die ERBA, Gartenstadt, Bamberg-Ost, der Hain, am Laubanger und das Berggebiet (siehe Abbildung HIER ABBILDUNG). Die Entscheidung über die Grenzen und der Gebiete erfolgt dabei durch die in der Literatur gegebenen Charakteristika, welche zum Definieren von LCZ notwendig sind \cite{oke2004initial, stewart2012local}. % Prof. Foken: Stadt gliedert sich in diese und jene Zone ein etc.

\subsection{Identifikation der Stakeholder}
\label{sec:stakeholder}

\paragraph{Der Bürgerverein Bamberg Mitte e.V.}

\paragraph{Die Domänenexperten} abc 123 test

\paragraph{Der Lehrstuhl für Informatik, insbesondere Mobile Softwaresysteme/Mobilität der Universität Bamberg}

\subsection{Das Bamberger Klimamessnetz als Grundlage der Sensordaten}

\subsection{Die Netatmo API --- eine Schnittstelle zwischen Sensordaten und Stakeholder}

\section{Definition der Anforderungen}

\section{Use Cases}