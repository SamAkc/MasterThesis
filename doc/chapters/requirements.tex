\chapter{Anforderungsanalyse} % Kapitel zu der Anforderungsanalyse
\label{sec:requirements}
% Wie ist die Ausgangslage (in Bamberg)? Wie werden/wurden die Anforderungen aufgestellt, 
% wer sind die Stakeholder und warum? Dann Anforderungen definieren und priorisieren
Bevor mit der eigentlichen Implementierung begonnen werden kann, ist es notwendig gewesen, die Frage nach den Anforderungen zu klären. Der erste Schritt 
hierzu besteht daraus, die Ausgangslage in Bamberg zu untersuchen, um darauf aufbauend die Stakeholder zu identifizieren, da diese den Ursprung der Anforderungen 
darstellen. Nachdem das Aufstellen der Anforderungen erfolgreich gewesen ist, können diese im Anschluss analysiert, also priorisiert, auf ihre Validität und im Anschluss 
auf ihre Realisierbarkeit überprüft werden. Basierend auf den Anforderungen wird dann ein Entwurf erstellt, der die Grundlage für die Implementierung darstellt. Mithilfe von 
Use Cases wird das entworfene Werkzeug dann auf seine Funktionalität hin überprüft.

\section{Ausgangslage in Bamberg}
\label{sec:ausgangslage}
Bamberg\sidenote{https://www.stadt.bamberg.de/Unsere-Stadt/Stadtinfo/} ist eine kreisfreie Stadt in Bayern im Regierungsbezirk Oberfranken mit einer Einwohnerzahl von knapp 80.000 Einwohnern 
(Stand: Dezember 2022). % Prof. Foken: Stadt gliedert sich in diese und jene Zone ein etc.

\subsection{Identifikation der Stakeholder}
\label{sec:stakeholder}
% BVM, Domänenexperten und MOBI-Lehrstuhl

\subsection{Das Bamberger Klimamessnetz als Grundlage der Sensordaten}

\subsection{Die Netatmo API - eine Schnittstelle zwischen Sensordaten und Stakeholder}

\section{Definition der Anforderungen}

\section{Use Cases}