\chapter{Anforderungsanalyse} % Kapitel zu der Anforderungsanalyse
% Wie ist die Ausgangslage (in Bamberg)? Wie werden/wurden die Anforderungen aufgestellt, 
% wer sind die Stakeholder und warum? Dann Anforderungen definieren und priorisieren
Bevor mit der eigentlichen Implementierung begonnen werden kann, ist es notwendig gewesen, die Frage nach den Anforderungen zu klären. Der erste Schritt 
hierzu besteht daraus, die Ausgangslage in Bamberg zu untersuchen, um darauf aufbauend die Stakeholder zu identifizieren, da diese den Ursprung der Anforderungen 
darstellen. Nachdem das Aufstellen der Anforderungen erfolgreich gewesen ist, können diese im Anschluss analysiert, also priorisiert, auf ihre Validität und im Anschluss 
auf ihre Realisierbarkeit überprüft werden.

\section{Ausgangslage in Bamberg}
Die Stadt Bamberg

\subsection{Identifikation der Stakeholder}
% BVM, Domänenexperten und MOBI-Lehrstuhl

\subsection{Das Bamberger Klimamessnetz als Grundlage der Sensordaten}

\subsection{Die Netatmo API - eine Schnittstelle zwischen Sensordaten und Stakeholder}

\section{Definition der Anforderungen}

\subsection{User Story \#1}

\subsection{User Story \#2}

\section{Entwurf}

\section{Use Cases}