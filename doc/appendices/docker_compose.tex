\chapter{Docker Compose}
\label{appendix:docker_compose}
Im folgenden Abschnitt wird das Docker Compose File aufgeführt, welches für die Entwicklung der Anwendung verwendet wird. Es ist zu beachten, dass die Anwendung nicht für den produktiven Einsatz gedacht ist und daher nicht für die Verwendung in einem Docker Swarm Cluster optimiert ist. Die definierten Services gliedern sich dabei ein wie folgt:

\begin{itemize}
    \item \textbf{PostgreSQL:} Datenbank für die Speicherung der Messwerte
    \begin{verbatim}
        postgresql:
        container_name: postgresql
        image: postgres:15
        restart: always
        environment:
          POSTGRES_DB: db
          POSTGRES_USER: samet
          POSTGRES_PASSWORD: admin
        ports:
          - "5432:5432"
        volumes:
          - postgresql:/var/lib/postgresql/data
          - ./db/data:/data
          - ./db/init.sql:/docker-entrypoint-initdb.d/init.sql
    \end{verbatim}
    \item \textbf{Grafana:} Visualisierung der Messwerte
    \begin{verbatim}
        grafana:
        container_name: grafana
        image: grafana/grafana:latest
        user: "0:0"
        environment:
          GF_DATABASE_TYPE: postgres
          GF_DATABASE_HOST: postgresql:5432
          GF_DATABASE_NAME: db
          GF_DATABASE_USER: samet
          GF_DATABASE_PASSWORD: admin
          GF_DATABASE_SSL_MODE: disable
          GF_SECURITY_ALLOW_EMBEDDING: true
        restart: unless-stopped
        depends_on:
          - postgresql
        ports:
          - 3000:3000
        volumes:
          - grafana:/var/lib/grafana
          - ./grafana/provisioning:/etc/grafana/provisioning
    \end{verbatim}
    \item \textbf{Web:} Flask Webserver für die Bereitstellung der Messwerte
    \begin{verbatim}
        web:
        container_name: flask-webserver
        image: python:3.8-slim
        ports:
          - "5000:5000"
        volumes:
        - ./web:/app
        working_dir: /app
        command: > 
          /bin/sh -c "pip install -r requirements.txt && flask --app index run --host=0.0.0.0"
        depends_on:
          - grafana
    \end{verbatim}
\end{itemize}

Im Anschluss müssen die Volumes definiert werden:
\begin{verbatim}
    volumes:
      postgresql:
        driver: local
      grafana:
        driver: local
\end{verbatim}